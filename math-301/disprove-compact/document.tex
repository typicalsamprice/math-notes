% -*- compile-command: "latexmk -pdf document.tex" -*-
\documentclass{article}

\input{template}

%\usepackage{lmodern}

\author{Sam Price}
\date{}
\title{Preview Assignment 6\\\Large{Question 5}}

\newcommand{\Rt}[1]{\dfrac{1}{10^{#1}}}

\begin{document}

\maketitle

Let $S = \Q \intersect [0, 1]$. Is $S$ compact? (Allowed to use Heine-Borel)
\vspace{0.5cm}

No, $S$ is not compact. We aim to show this by constructing an open cover which does not admit a finite subcover.

Firstly, assume every element in $S$ is written in lowest terms. That is,
\[ S = \set*{\frac{p}{q} : 0 \le p \le q, \gcd(p, q) = 1 }. \]

Let the cover of $S$
\[ \sC = \set*{\parens*{\frac{a}{b} - \Rt{b}, \frac{a}{b} + \Rt{b}}}_{\frac{a}{b} \in S} \]
be the central object of this proof.

Unrelated to the actual proof, but the bounds on the sum of all were a little bit off, in that
\[
  \sum_{C \in \sC} \ell(C) = \sum_{k = 1}^{\infty}\frac{2\varphi(k)}{10^{k}}
  < \frac{2}{10} + \sum_{k = 2}^{\infty}\frac{2(k - 1)}{10^{k}}
\]
with $\varphi$ being Euler's totient function.


\section{Definitions \& Notation}
In this article, let $\parens*{\frac{p}{q}}_{\gamma} = q$, for any rational $p/q$ in canonical form.

For any rational $r \in S$, let us denote
\[ U_{r} = \parens*{r - \Rt{r_{\gamma}}, r + \Rt{r_{\gamma}}} \]
and note that $U_{r} \ne \emptyset$ for any $r$.

\section{Proof}

Suppose for the sake of contradiction that there exists some finite subcover $\sU \subset \sC$ for $S$.
Therefore, let $R$ be the (ordered) set of rationals indexing $\sU$:
\[ R = \set*{r_{1}, r_{2}, \ldots, r_{n}}. \]
Define $\alpha = 1/\!\sqrt{3}$ as the (positive) root of
\[ p(x) = 3x^{2} - 1, \]
which lands within $(0, 1)$.
Now, by Liouville's Theorem, take $1/M = 1/(6 + 6\alpha)$ and create $A = 1/M - 10^{-10} < 1/M$
and say that for all $v \in S$,
\[ \abs{\alpha - v} > \frac{A}{v_{\gamma}^{2}} > \frac{1}{9.6v_{\gamma}^{2}} > \frac{1}{10^{v_{\gamma}}} \implies \alpha \notin U_{v}. \]
Hence, $\alpha \notin U_{r}$, for every $r \in R$.
However, this also means that in $R$, there is some ``closest''
(for its matching interval, whether the infimum or supremum) $r_{i}$.
This distance can be encoded by
\[ d = \min\set*{\min\set*{\abs*{\alpha - \inf U_{r_{i}}}, \abs*{\alpha - \sup U_{r_{i}}}}_{r_{i} \in R}} > 0. \]
If we let $d > \eps > 0$, there is for all $r_{i} \in R$
\[
  \abs*{\alpha - r_{i}} - \eps > \Rt{(r_{i})_{\gamma}}.
\]
Hence, we find that
\[
  \parens*{\alpha - \eps, \alpha + \eps} \intersect \parens*{\bigcup_{r \in R}U_{r}} = \emptyset.
\]
and therefore that because $(\alpha - \eps, \alpha + \eps)$ is nonempty,
there must be an infinite number of rationals uncovered under $\sU$.

Thus, $S$ is not compact.

\qed{}

% Jotting down ideas:
% Show that no finite subcover $\sU$ covers [0, 1] in the reals
% - Perhaps via showing that if $q$ is the max, then there are reals covered
%   in some group of open intervals (by Hurwitz) and that those intervals summed
%   make up less than 1.
%   Inspired by: SoME3 video "What Happens If We Add Fractions Incorrectly?"
%   Although I want to be vehement about this NOT being my first exposure to Farey sequences and similar arguments,
%   since I found those a couple days ago.
%   Also, I found the upper bound for $n$ BEFORE learning of Farey sequences, and I am very proud of that.

% Another, more brilliant idea:
% Show that C does NOT cover [0, 1] in the first place
% Then, there exists some real number \alpha not covered by a finite subcover U
% However, since by Diophantine Approximation we can find arbitrarily close p/q to \alpha
% Thus, if S is covered by U, then \alpha must be in the boundary of the union of all U
% However, the \partial\mathcal{U} is only rational points. Thus, a contradiction.

\end{document}
