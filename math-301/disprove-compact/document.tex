% -*- compile-command: "latexmk -pdf document.tex" -*-
\documentclass{article}

\input{template}

%\usepackage{lmodern}

\author{Sam Price}
\date{}
\title{Preview Assignment 6\\\Large{Question 5}}

\newcommand{\Rt}[1]{\dfrac{1}{10^{#1}}}

\begin{document}

\maketitle

Let $S = \Q \intersect [0, 1]$. Is $S$ compact? (Allowed to use Heine-Borel)
\vspace{0.5cm}

No, $S$ is not compact. We aim to show this by constructing an open cover which does not admit a finite subcover.

Firstly, assume every element in $S$ is written in lowest terms. That is,
\[ S = \set*{\frac{p}{q} : 0 \le p \le q, \gcd(p, q) = 1 }. \]

Let the cover of $S$
\[ \sC = \set*{\parens*{\frac{a}{b} - \Rt{b}, \frac{a}{b} + \Rt{b}}}_{\frac{a}{b} \in S} \]
be the central object of this proof.

Unrelated to the actual proof, but the bounds on the sum of all were a little bit off, in that
\[
  \sum_{C \in \sC} \ell(C) = \sum_{k = 1}^{\infty}\frac{2\varphi(k)}{10^{k}}
  < \frac{2}{10} + \sum_{k = 2}^{\infty}\frac{2(k - 1)}{10^{k}}
\]
with $\varphi$ being Euler's totient function.


\section{Definitions \& Notation}
In this article, let $\frac{p}{q}_{\gamma} = q$, for any rational $p/q$ in canonical form.

For any rational $r \in S$, let us denote
\[ U_{r} = \parens*{r - \Rt{r_{\gamma}}, r + \Rt{r_{\gamma}}} \]
and note that $U_{r} \ne \emptyset$ for any $r$. We will also use the fact that
\[ \sup U_{r} - \inf U_{r} = \frac{2}{10^{\gamma(r)}}. \]

Let us also use the notation $F_{n}$ for \emph{Farey sequences}, or the set of all rationals
on the unit interval (also $S$, in this case) with denominators less than or equal to $n$ (with $n \in \N$, naturally).
For instance, we see
\[ F_{3} = \set*{\frac{0}{1}, \frac{1}{3}, \frac{1}{2}, \frac{2}{3}, \frac{1}{1}}. \]
I derived this myself, but also found out later I was right in that
\[ \abs{F_{n}} = \Phi(n) + 1 \]
with $\Phi$ being the summatory totient function.

\section{Lemmas}

\begin{lem}
  If $r, s \in S$ and $r \in U_{s}$, then $r_{\gamma} > s_{\gamma}$.
\end{lem}
\begin{proof}
  Let $r, s \in S$ such that $r \in U_{s}$. For the sake of contradiction, suppose $r_{\gamma} \le s_{\gamma}$.
  Now, since $r \in U_{s}$, we know
  \[ s - \Rt{s_{\gamma}} < r < s + \Rt{s_{\gamma}}. \]
  Since $r_{\gamma} \le s_{\gamma}$, we know $2/10^{s_{\gamma}} < 1/10^{r_{\gamma}}$. Therefore, $U_{s} \subset U_{r}$.
\end{proof}

\section{Proof}
Suppose there exists a finite subcover $\sU \subset \sC$ for $S$ with $n$ elements
\[ r_{1}, r_{2}, \ldots, r_{n} \]
with the relationship
\[ 0 = r_{1} < r_{2} < \cdots < r_{n} = 1. \]
Let $q$ denote the maximum of the denominators of all $r_{i}$.
As such, we note
\[ \set*{r_{1}, \ldots, r_{n}} \subseteq F_{q}. \]

For the sake of contradiction, suppose there exists some $r \in S$ such that
\[ \abs{r - r_{i}} \ge \Rt{r_{i\gamma}} \ge \Rt{q}. \]



\qed{}

% Jotting down ideas:
% Show that no finite subcover $\sU$ covers [0, 1] in the reals
% - Perhaps via showing that if $q$ is the max, then there are reals covered
%   in some group of open intervals (by Hurwitz) and that those intervals summed
%   make up less than 1.
%   Inspired by: SoME3 video "What Happens If We Add Fractions Incorrectly?"
%   Although I want to be vehement about this NOT being my first exposure to Farey sequences and similar arguments,
%   since I found those a couple days ago.
%   Also, I found the upper bound for $n$ BEFORE learning of Farey sequences, and I am very proud of that.

% Another, more brilliant idea:
% Show that C does NOT cover [0, 1] in the first place
% Then, there exists some real number \alpha not covered by a finite subcover U
% However, since by Diophantine Approximation we can find arbitrarily close p/q to \alpha
% Thus, if S is covered by U, then \alpha must be in the boundary of the union of all U
% However, the \partial\mathcal{U} is only rational points. Thus, a contradiction.

\end{document}
