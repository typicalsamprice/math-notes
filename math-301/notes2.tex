% -*- compile-command: "latexmk -pdf notes2.tex" -*-
\documentclass{article}

\let\boxedthm1
\input{template}
\usepackage{cleveref}

\newcommand{\todo}{\textbf{TODO}: }

\title{Intro to Real Analysis}
\author{Sam Price}

\begin{document}

\maketitle

% Want to:
% Show | x - y | < \eps => x = y

\section*{Review Some Stuff}

\subsection*{Absolute Value}
As a reminder, we know that the absolute value is:
\[ \abs{x} = \begin{cases} x & x \ge 0\\ -x & x < 0. \end{cases} \]

This means that $\abs{x} < a$ is the same as $-a < x < a$, which is useful if one ever needs to swap absolute values in and out.

Indeed, this function has a few nice properties, where for all $x, y, z \in \R$:
\begin{itemize}
  \item $\abs{xy} = \abs{x}\abs{y}$
  \item $\abs{x + y} \le \abs{x} + \abs{y}$, which is sometimes called the Triangle Inequality.
        This is invaluable when trying to squeeze intermediate steps into the most
        unwieldy $<$-chain you've ever constructed.
  \item $\abs{x - y} = \abs{y - x}$
  \item $\abs{x^{z}} = \abs{x}^{z}$, somewhat implied by the product law above.
        Note that $x = 0$ will probably not play nice with $z < 0$, since dividing by
        zero can be a bit of a pain.
\end{itemize}

\subsection*{Sets}
In this, we will let $\N \subset \Z \subset \Q \subset \R$ be their normal designations, and $\N$ will \emph{start with} 1, not 0.
I will also do my best while writing to make sure to use $A \subset B$ \emph{only} when I can justify $A \ne B$, and use $A \subseteq B$ normally.

\section*{Suprema and Infima}

\section*{Sequences}

Sequences are the foundation for everything in analysis --- or at least Spence will make it seem that way.
I believe Fall~2024 actually has Hough teaching this course (Math 301) but I could be mistaken.
In any case, sequences are invaluable and in fact it is very possible you could reach what
Terrence Tao refers to as some ``post-rigor'' stage with sequences: using them informally and heuristically
while being able to adapt them formally when needed.

\begin{defn}[Sequences]\label{defn:sequence}
  A sequence $(a_{n})_{n \in \N}$ is an infinite ordered list of elements of some set, in this course $\R$.
  This can be written as simply $(a_{n})$ or $a_{n}$, depending on context the latter may refer to a single specific element of $(a_{n})$.
  Sometimes, one might see a sequence indexed by a set other than $\N$, such as $\R$. In general, they are written as
  \[ (a_{i})_{i \in I} \]
  with $I$ being some indexing set. In this, unless stated otherwise, sequences are always infinite.
\end{defn}

\begin{defn}[Sequence Convergence]\label{defn:convergence}
  A sequence $(a_{n})$ can be thought of as \emph{converging} to a value $a \in \R$ if it is always arbitrarily close
  to $a$ after a ``certain point''. This is denoted as $(a_{n}) \to a$ and means that for every $\eps > 0$,
  there is some natural number $N$ such that for $n > N$, there must be
  \[ \abs{a_{n} - a} < \eps. \]
  This can be thought of similarly to the (perhaps overly orchestrated) criteria that $x = y$.
  Note also that given some $N$ this holds for, any $M > N$ also will clearly hold for a given $\eps$.

  \todo Illustration, in particular showing that for $n \le N$, $\abs{a_{n} - a} < \eps$ \emph{may} hold,
  but not necessarily.
\end{defn}

\begin{theorem}[Monotone Convergence Theorem]\label{thm:mct}
  Every monotone (non-decreasing or non-increasing) and bounded sequence $(a_{n})$ is convergent.
\end{theorem}

Before we get to the fun part, let us (formally) define what a subsequence is.
Intuitively, it's just a some part of a sequence ``in order'', which is absolutely true.
To nail down intuition, we define it as:

\begin{defn}[Subsequence]\label{defn:subsequence}
  A subsequence of $(a_{k})$ is a sequence $(a_{n_{k}})$ where $(n_{k})$
  is a strictly increasing sequence of natural numbers.
\end{defn}

This means that one example is $(n_{k}) = (1, 3, 17, 1001, \ldots)$, but a non-example
could be something like $(n_{k}) = (1, 3, \bm{3}, 17, \bm{14}, \ldots)$ which is invalid both because 3 \emph{repeats} and
14 is \emph{non-increasing} (i.e.\ both indices' elements do not grow by one or more when compared to the previous.)

The following theorem is incredibly important, and is attributed to Karl Weierstrass and Bernard Bolzano.
This will be proven in class, but is super important --- I recommend sketching out the diagram yourself to visualize why this
must be true.
\begin{theorem}[Bolzano-Weierstrass]\label{thm:bolzano-weierstrass}
  Every bounded sequence $(a_{n})$ has a convergent subsequence.
\end{theorem}
\begin{proof}
  Let $(a_{n})$ be a bounded sequence below and above by $m, M \in \R$ (that is, for all $n$, $m \le a_{n} \le M$.)
\end{proof}

\begin{defn}[Cauchy Sequence]\label{defn:cauchy}
  A sequence $(a_{n})$ is called \emph{Cauchy} (after Augustin Louis Cauchy) if
  for all $\eps > 0$, there is some natural number $N$ such that for all $m, n > N$ there is
  \[ \abs{a_{m} - a_{n}} < \eps. \]
  This is similar to simply \emph{convergence}, but is different in the sense that
  instead of measuring how close the sequence gets to some ``final'' value, it measures
  how terms get close \emph{and then stay close}.
  In many proofs, one will assume without loss of generality that $m > n$ or vice versa.
\end{defn}

\begin{theorem}\label{thm:cauchy-iff-convergent}
  A sequence $(a_{n})$ is Cauchy if and only if it is convergent.
\end{theorem}
\begin{proof}
  ($\implies$) Let $\eps > 0$, and note that $\eps/2 > 0$ as well. Then, there exists some $N$ such that for $m > n > N$
  \[  \]
\end{proof}

\subsection*{Series}

Your favorite friend from Calculus II\@! Unfortunately, or perhaps fortunately, we get to formulate series using sequences.
However, our series is thought of as
\[ \sum_{n = 1}^{\infty}a_{n} \]
for some sequence $(a_{n})$, but that isn't particularly useful as of yet.
Instead, we consider the sequence of \emph{partial sums} $(s_{n})$, where
\[ s_{n} = \sum_{k = 1}^{n}a_{k}. \]
This allows us to talk about series convergence, everyone's favorite thing to check for. Similar to sequence convergence,
series convergence looks like this:
\begin{defn}[Series Convergence]
  A series $\sum a_{k} = S$ if:
  \[ \forall \eps > 0\, \exists N \in \N\, \forall n > N, \abs{s_{n} - S} < \eps.  \]
\end{defn}


\section*{Topology of $\R$}

\subsection*{Closed and Open Sets}

\subsection*{Compactness}

\subsubsection*{Why Compactness Matters}

\section*{Limits, Continuity and Differentiability}

\section*{Integrals}

\end{document}
