\documentclass{article}

\input{template}

\author{Sam Price}
%\date{}
\title{Homework \#1}

\begin{document}

\maketitle

\begin{enumerate}

        \item Get all these suprema and infima done!
        \begin{enumerate}

            \item This will be for both $\OpenInt{0, 1}$ and $\ClosedInt{0, 1}$.\\
                Let $A = \OpenInt{0, 1}$ and $B = \ClosedInt{0, 1}$. A viable lower bound for
                both sets is 0, since $0 \le x$ for all $x \in A$ and $x \in B$.
                If 0 is \textit{not} the infimum, then there exists some $\alpha > 0$ such that $\alpha \le x$ for all $x$ in $A$ and $B$.
                However, for any $\alpha > 0$, we know $0 < \alpha/2 < \alpha$ and $\alpha/2 \in A$. Thus, there cannot be some $\alpha > 0$
                such that $\inf A = \alpha$. This is also true of $\inf B$, which is trivially the same where 0 takes the equality branch.

                The suprema for $A$ and $B$ is similarly checked. We know by simple checking that 1 is \textit{an} upper bound for both sets.
                Suppose that $\sup A = \alpha < 1$ and that $\eps = 1 - \alpha > 0$.
                This cannot be an upper bound however, since $\alpha + \eps/2 \in A$ and is greater than $\alpha$.
                Thus, $\sup A = \sup B = 1$.

                To justify doing both at once, we can see that $A \union \set{0, 1} = B$, or that $\set{0, 1} \subset B$.
                This means that we aren't adding any numbers ``in between'' other than those explicitly listed, which won't disrupt the integrity
                of the justifications.

          \item $A = \set{2/n : n \in \N}$.\\
                This one is similar to the one in class, but each element is twice its previous ``corresponding'' element.
                The supremum is trivially 2, since $2/x \to 0$ as $x \to \infty$ and $2 \le 2$.

                For the greatest lower bound, we know 0 is a possibility since every element of $A$ is positive.
                Let us assume there is some greater lower bound however, such that $\inf A = \eps > 0$.
                Since $\eps > 0$ however, we can find some $n \in \N$, $1/n < \eps/2 \implies 2/n < \eps$.
                This shows that there is an element in $A < \eps$, meaning there cannot be a greater lower bound and that $\inf(A) = 0$.

          \item $A = \OpenInt{0, 1} \union \set{2}$.\\
                This has the exact same infimum as described in $(a)$, so the point will not be belabored.
                For the supremum, we know $2 \ge 2$, so 2 is a candidate, and is a clear upper bound for $\OpenInt{0, 1}$.
                The fact there are no lesser upper bounds is left as a check for the reader,
                since it seems clear at this point $\sup(A) = 2$.

          \item $A = \set*{\frac{1}{n} - \frac{2}{m} : n, m \in \N}$.\\
                The upper bound of 1 is a clear contender for $\sup A$, since $2/m > 0$ and $1/n \le 1$, we know 1 is greater than
                every element in $A$. Suppose the true supremum of $A$ is some $\alpha = 1 - \eps$ where $\eps > 0$.
                We could then take $n = 1$, and find some $m$ such that $2/m < \eps$. Since this is within $A$ but greater than $\alpha$,
                such an $\alpha \ne \sup A$. Thus, we can say $\sup A = 1$.

                One valid value for $\inf A$ is -2. Such a number cannot be achieved (or exceeded), but
                can be clearly approached with $\lim_{n \to \infty} (1/n - 2) = -2$ and $m = 1$.
                Suppose $\inf A > -2$, and that such an infimum is $\alpha$ and that $\eps = -2 - \alpha > 0$.
                By letting $m = 1$ (to maximize said term), we can find a sufficiently large $n$
                such that $0 < 1/n < \eps$. By the property of the denseness of $\Q$ (and $\R$), there exists such an $n \in \N$.
                Therefore, $1/n - 2 < \alpha$ but is still within $A$. Thus, we can say $\inf A = -2$.
        \end{enumerate}

  \item Assume $A \subseteq B$ are both bounded above (sufficiently, $B$ is). Prove that $\sup A \le \sup B$.
        \begin{proof}
          Suppose not, and that $\sup B < \sup A$. Thus, we can say there exists some $x \in A$ such that $\sup B < x \le \sup A$.
          However, since $A \subseteq B$, we know that $x \in B$, and therefore $\sup B < x$ is a contradiction to our original assumption.
          Thus, $A \subseteq B \implies \sup A \le \sup B$.
        \end{proof}

  \item Suppose $A \subset \R$ has a maximal element $M \in A$. Likewise, suppose $B \subset R$
        has a minimal element $m \in B$.
        \begin{enumerate}
          \item Prove that $\sup A = M$.
                \begin{proof}
                  Clearly, $\sup A \not > M$, since otherwise $M < \sup A$ is a lesser upper bound.
                  Suppose not, and that $\sup A < M$. Then, $M \in A$ would be impossible, as $M > \sup A \ge M$ is a contradiction.
                  Therefore, $\sup A = M$.
                \end{proof}
          \item Prove that $\inf B = m$.
                \begin{proof}
                  Clearly, $\inf B \not < m$, since otherwise $m > \inf B$ is a greater lower bound.
                  Suppose not, and that $\inf B > m$. Then we would find that $m < \inf B \le m$, which is an impossibility.
                  Therefore, $\inf B = m$.
                \end{proof}
        \end{enumerate}

  \item Suppose $A$ is a nonempty set containing some finite number of elements. Prove that $A$ has a maximal element,
        and that $\max(A) \in A$.
        \begin{proof}
          Let $A$ contain 1 element. Since there is only one element, said element is clearly maximal \textit{and} contained
          within $A$. Let us assume then, that if $\abs{A} = n \in \N$, then it has some maximal element $a \in A$.

          Now for the case of $\abs{A} = n + 1$, we have two scenarios: either the maximal element is within the first $n$
          elements, in which case by our hypothesis said maximal element $a \in A$, or it is the $n$+1\textsuperscript{th} element, where
          it is greater than every other $a \in A$ and thus is the maximal element contained within $A$.
        \end{proof}

  \item Suppose $A \subset \R$ is bounded above and $c \in \R$. Define $c + A = \set{c + a : a \in A}$
        and $cA = \set{ca : a \in A}$.
        \begin{enumerate}
          \item Prove $\sup(c + A) = c + \sup(A)$.
                \begin{proof}
                  Let $\beta = \sup(A)$. Notice that for $c + A$, $c + \beta$ is an upper bound, since
                  for any $a \in A$, $a \le \beta \implies a + c \le \beta + c$.
                  Since $\beta - \eps < a$ for some $a \in A$ given any $\eps > 0$,
                  it follows that $\beta - \eps + c < a + c$. Thus, there cannot be a lesser upper bound
                  for $\sup(c + A)$, and $\sup(c + A) = c + \sup(A)$.
                \end{proof}

          \item Discern the necessary and sufficient (if and only if) conditions such that $\sup(cA) = c\sup(A)$.
                Also give an example of some $c, A$ such that $\sup(cA) \ne c \sup A$.

                The condition holds if $c \ge 0$ or $\abs{A} \le 1$.
                \begin{proof}
                  Firstly, $\abs{A} \le 1$ must imply $c\sup(A) = \sup(cA)$.
                  This is because if $A$ is empty, then there is no supremum for $A$ nor $cA$.
                  So, consider the case $\abs{A} = 1$.
                  Since the only element of $A$ \textit{is} the supremum, it must be equal to its multiple of $c$
                  no matter which order the ``operations'' are applied.
                  That is, $c \sup(\set{x}) = cx = \sup(\set{cx})$.\\

                  Now, $c \ge 0$ must also imply the identity is true. Let $\beta = \sup(A)$, and consider the set $cA$.
                  The value $c\beta$ \textit{is} an upper bound, since for each element $ca$ in $cA$, $c\beta$ is greater.
                  This is because $c \ge 0$, and $a \le \beta \implies ca \le c\beta$.\\

                  Considering any lower upper bounds, let $\eps > 0$ and $c > 0$ since the case $c = 0$ is trivial.
                  The value $c\beta - \eps$ cannot be an upper bound of $cA$ however,
                  since for some $a \in A$:
                  \begin{equation*}
                    \frac{c\beta - \eps}{c} = \beta - \frac{\eps}{c} < a.
                  \end{equation*}
                  With the rightmost inequality being true by the definition of the supremum.
                  This means that $ca > c\beta - \eps$, and is a contradiction to the definition of the supremum
                  and our assumption of some lesser upper bound,
                  therefore $c\sup(A) = c\beta = \sup(cA)$.\\

                  Now, to prove the other direction let us consider the contrapositive.
                  Let $c < 0$ and $\beta = \sup(A)$ where $\abs{A} > 1$.
                  Take two elements $x, y \in A$ such that $x < y \le \beta$.
                  Now, $x < y \le \beta \implies c\beta \le cy < cx$ since $c < 0$.
                  Thus, $c\beta < cx$, and therefore $c\sup(A) = c\beta \ne \sup(cA)$.
                  More specifically, $c\sup(A) < \sup(cA)$ in this case.
                \end{proof}


                Counterexample: $A = \set{1, 2, 3}$ and $c = -1$, where $\sup(cA) = -1$ and $c \sup A = -3$.
        \end{enumerate}


  \item For $A \subseteq \R$, we denote $-A$ to be the set
        \begin{equation*}
          -A = \set{ -x: x \in A }.
        \end{equation*}

        Suppose $A \ne \emptyset$ and that $A$ is bounded below.
        Prove that $-A \ne \emptyset$, $-A$ is bounded above,
        and that $\sup(-A) = -\inf(A)$.\\

        The first requirement is obvious, and won't be put in its own proof-box.
        Simply put, there is the trivial bijection between $A$ and $-A$ (the negation of each element, and 0 mapped to itself) and thus
        $0 < \abs{A} = \abs{-A} > 0$.
        (AN:\ I feel very satisfied knowing this is the formal definition of equal cardinality as well.
        After: This felt much cooler before you went over the definition in class, but hey-ho that's how it goes.)

        \begin{proof}[Upper bound of $-A$]
          Suppose for the sake of contradiction that $-A$ is \textit{not} bounded above.
          This means that for any $n \in \N$, an element $a$ can be found in $-A$ such that $a > n$.
          Thus, when we take the set $-(-A)$, this gives us
          \begin{equation*}
            -(-A) = \set{ -(-x) : x \in A } = \set{ x : x \in A } = A.
          \end{equation*}

          Since $-A$ is not bounded above, we can find arbitrarily large (negative) values in $-(-A)$ (and therefore $A$). However, this
          is a contradiction since we know $A$ is bounded below.
        \end{proof}

        \begin{proof}[Value of $\sup(-A)$]
          Firstly, suppose $\beta = \sup(-A)$ and that $\gamma = \inf(A)$.
          We (now) know that both of these values exist, since $A$ is bounded below and $-A$ above.

          Note first that $-\gamma$ is a known upper bound, since if $\gamma \le a$ for all $a \in A$,
          this also means that $-\gamma \ge -a \implies -\gamma \ge \beta$. Assume that $\beta < -\gamma$.

          However, $\beta$ is then not a possible upper bound since for any $\eps > 0$, $\gamma + \eps > a$ for some $a \in A$.
          Following this, $-\gamma - \eps < a$ for some $a \in -A$ and thus choosing $\eps = -\gamma - \beta$:
          \begin{equation*}
            \beta = -\gamma - \eps < a.
          \end{equation*}

          However, $\beta < a$ means that $\beta$ is \textit{not} an upper bound on $-A$, and
          therefore we can conclude that $-\gamma$ is an upper bound and in fact the least upper bound, meaning $\beta \ge -\gamma$,
          and then that $\sup(-A) = -\inf(A)$.
        \end{proof}

  \item Let $x, y \in \R$. Prove that if $x < y + \eps$ for all $\eps > 0$, then $x \le y$.
        Conclude that if $\abs{x - y} < \eps$ for all $\eps > 0$, then $x = y$. Does the proof work if $<$ is replaced by $\le$?\\

        To answer the second point first, with the initial proof being sort of lemma'd in, we can say \textit{yes}, it in fact does work.
        \begin{proof}
          Let $x, y \in \R$ such that $\abs{x - y} \le \eps$ for all $\eps > 0$. Suppose that $\abs{x - y} = \eps$ for some such $\eps$.
          Thus, we could say that
          \begin{equation*}
            \abs{x - y} = \eps > \frac{\eps}{2} \implies \abs{x - y} > \frac{\eps}{2}.
          \end{equation*}

          This however would be a contradiction to the \textit{all} quantifier for $\eps$ since $\eps/2 > 0$ as well.
          As such, $<$ is equivalent to $\le$ since equality can never happen.
        \end{proof}

        Now, the first proof and its corollary.
        \begin{proof}
          Let $x, y \in \R$ such that $x < y + \eps$ for all $\eps > 0$.
          Suppose that $x > y$.
          Thus, when $\eps = y - x$, $x = y + \eps$. However, $x = y + \eps$ is a contradiction to $x < y + \eps$,
          since equality disobeys the $<$ relation.
          Thus, $x \not> y$, and in fact $x \le y$.
        \end{proof}

        \begin{proof}[Corollary]
          Let $x, y \in \R$ such that $\abs{x - y} < \eps$ for all $\eps > 0$.

          To first take $x - y \ge 0$, then $\abs{x - y} = x - y < \eps$. By the above theorem,
          we see that since $x < y + \eps$, then $x \le y$.

          Similarly, when $x - y < 0$, then $\abs{x - y} = y - x$. In the same style of argument,
          notice that this implies $y \le x$. The only time $x \le y$ and $x \ge y$ is when $y = x$,
          and thus $\abs{x - y} < \eps$ for all $\eps > 0$ necessitates that $x = y$.
        \end{proof}
\end{enumerate}

\end{document}
