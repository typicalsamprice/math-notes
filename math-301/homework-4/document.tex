% -*- compile-command: "latexmk -pdf document.tex" -*-
\documentclass{article}

\input{template}

\author{Sam Price}
\date{}
\title{Homework 4\\\Large{Take 2, Restarting Because Computers Suck}}

\begin{document}

\maketitle

This had to be entirely restarted, because my computer had a stork. Birds are annoying, and I hope it lasts me through the semester at least.
The particularly tedious parts may be left for later since this is really gonna get to me if it happens again (frequent commits, I hope?)

\begin{enumerate}

  % One
  \item Tedious, TBD


  % Two
  \item Fix $\eps = 1$. Then, we see $f(x) = 3$ for $x < 3$ and $f(x) = 5$ for $x > 5$. Thus, we note for all $\delta > 0$
        \[ 0 < \abs{x - 3} < \delta \implies \abs{f(x) - L} < 1. \]
        However, this means that
        \[ \abs{3 - L} < 1 \quad\text{and}\quad \abs{5 - L} < 1 \]
        which implies that $\abs{5 - 3} < 2$ which is clearly false.

  % Three
  \item Let $g(x) = x \cdot 1_{\Q}(x)$. Show that $g$ is continuous at $c$ iff $c = 0$.

        \begin{proof}
          $\implies$. Let $c \ne 0$ and fix $\eps = \abs{c}/2$. Assume first that $c \in \Q$.
          For all $\delta > 0$ we may find a suitably close irrational $x$ such that $\abs{g(x) - g(c)} = \abs{c} > \abs{c}/2$.
          Now, say $c \notin \Q$. Then, we may pick a rational $x$ suitably close to $c$ such that
          \[ \abs{g(x) - g(c)} = \abs{x} > \abs{c}/2. \]
          We may say the inequality holds because there are infinitely many $x = p/q$ such that
          \[ \abs*{c - \frac{p}{q}} < \frac{1}{q^{2}}
            < \frac{\abs{c}}{2}
            \implies x \in N_{\abs{c}/2}(c) \implies f(x) \in \parens*{ c - \frac{\abs{c}}{2}, c + \frac{\abs{c}}{2} }
          \]
          and thus $\abs{x} > \abs{c}/2 = \eps$.

          $\impliedby$. Let $c = 0$ and $\delta = \eps$. Then, we see
          \[ \abs{x} < \delta \implies \abs{g(x) - g(0)} = \abs{g(x)} \le \abs{x} < \delta = \eps. \]
        \end{proof}
  % Four
  \item Let $f \from \R \to \R$ be continuous, and 0 at rational numbers. Prove that $f(x) = 0$ for all $x \in \R$.

        \begin{proof}
          Suppose there is some $\alpha \notin \Q$ such that $f(\alpha) \ne 0$. Fix $\eps = \abs{f(\alpha)}$.
          For each $\delta > 0$ we see that
          \[ \abs{x - \alpha} < \delta \implies \abs{f(x) - f(\alpha)}. \]
          However, similar to before, we may pick some rational $x$ within $N_{\delta}(\alpha)$, and so find
          that
          \[ \abs{x - \alpha} < \delta \quad\text{and}\quad \abs{f(x) - f(\alpha)} = \abs{f(\alpha)} \not< \eps. \]
        \end{proof}
  % Five
  \item Also very tedious, but I am writing down the left- and right-hand limit definitions here still:

        \[\tag{left} \forall \eps > 0 \, \exists \delta > 0, 0 < c - x < \delta \implies \abs{f(x) - L} < \eps. \]
        \[\tag{right} \forall \eps > 0 \, \exists \delta > 0, 0 < x - c < \delta \implies \abs{f(x) - L} < \eps. \]



  % Six
  \item
        \begin{enumerate}

          \item Prove $h(x) = \sin(1/x)$ is continuous at $x \ne 0$.

                By Prop 6.20 $1/x$ is continuous since $x \ne 0$ and it is a division of $C^{0}$ functions.
                Similarly, by Prop 6.24 $h$ is then continuous since it is a \emph{composition} of continuous functions.

          \item Prove $h$ is discontinuous at $x = 0$.

                \begin{lem}\label{lem:sin-dense}
                  The set $\sin(\N)$ is dense in $[-1, 1]$.
                \end{lem}

                The proof of Lemma~\ref{lem:sin-dense} will be at the end of this whole Fest.

                \begin{proof}
                  Define the sequence $(a_{n})_{n \in \N}$:
                  \[ a_{n} = \frac{1}{n} \]
                It should be clear that $h(a_{n}) = \sin(\N)$ and that $a_{n} \to 0$.
                By Lemma~\ref{lem:sin-dense}, $\sin \N$ is dense, and so $h(a_{n})$ (as a whole sequence) is dense as well.
                Suppose for the sake of contradiction that $h(a_{n}) \to 0$ as well.
                Fix $\eps = 1/10$. Then, by assumption, there is some $M$ such that $\forall n > M$ we have
                \[ \abs{h(a_{n})} < \eps. \]
                Now, consider the set $S = \set{ \abs{h(a_{n})} \ge \eps : n \le M }$.
                Since $M$ is finite, so must $S$ be too (and we know it is nonempty as well because $\sin(2) > \sin(1) > 1/10$.)

                Thus, let $a = \max S$ and $b = \max S \setminus a$. By our choice of $a, b$ there is no value of $h(a_{k})$ between them,
                but this is contradictory to the fact that $h(a_{k})$ is dense. Hence, $h$ cannot be continuous at $x = 0$.
                \end{proof}

          \item Prove $g(x) = x \cdot h(x)$ is continuous at $x = 0$.

                \begin{proof}
                  Let $\eps > 0$ and pick $\delta = \eps$. Then, we see that
                  \[ 0 < \abs{x - 0} < \delta \implies \abs{g(x) - g(0)} = \abs{g(x)} \le \abs{x} < \delta = \eps. \]
                  This is true because $\sin(y) \in [-1, 1]$ and $\abs{x \sin y} \le \abs{x}$ then.
                \end{proof}

        \end{enumerate}

  % Seven
  \item Let $f \from \R \to \R = ax + b$ for some fixed real numbers $a$ and $b$. Prove $f$ is uniformly continuous.
        \begin{proof}
          Let $\eps > 0$, and pick $\delta = \eps/\abs{a}$. Then for $x, y \in \R$:
          \[ \abs{x - y} < \delta \implies \abs{ax + b - (ay + b)} = \abs{ax - ay} = \abs{a}\abs{x - y} < \delta\abs{a} = \eps. \]
        \end{proof}

  % Eight
  \item
        \begin{enumerate}

          \item Prove that if $f \from A \to \R$ is uniformly continuous and $A$ is bounded, then $f(A)$ is bounded.

                \begin{proof}
                  Let $f \from A \to \R$ be UC and $A$ be bounded by $a = \sup A, b = \inf A$.
                  Fix $\eps > 0$. Then, there is some $\delta > 0$ such that we can cover $A$ with $n$ many $\delta$-balls.
                  In each $\delta$-ball (denoted $\delta_{i}$ with center $c_{i}$), we have for $x \in \delta_{i}$
                  \[ \abs{x - c_{i}} < \delta \implies \abs{f(x) - f(c_{i})} < \eps. \]
                  In fact, we notice
                  \[ \abs{f(x)} = \abs{f(x) - f(c_{i}) + f(c_{i})} \le \abs{f(x) - f(c_{i})} + \abs{f(c_{i})} < \eps + \abs{f(c_{i})}. \]
                  Then, we see for $x \in \delta_{i}$
                  \[ \abs{f(x)} < \eps + \max\limits_{1 \le i \le n}\abs{f(c_{i})}. \]
                  Since $f$ is then bounded on each $\delta_{i}$ and there are finitely many, it is bounded by the maximum of them and therefore
                  $f(A)$ is bounded altogether.
                \end{proof}

          \item Both $\ln x$ and $1/x$ work, which are two sides of the same coin. Provided, of course, the coin is differential (or integral) calculus.

          \item Consider $\sin(1/x)$ with a fixed $\eps = 1/2$ and $x, y \approx 0$.

        \end{enumerate}

  % Nine
  \item Let $f, g$ be continuous on $[a, b]$ with $f(a) < g(a)$ and $f(b) > g(b)$. Prove $\exists c \in (a, b)$ such that $f(c) = g(c)$.

        \begin{proof}
          Define $h = f - g$. $h$ is also continuous, and $h(a) < 0 < h(b)$.
          By the intermediate value theorem we then know there is some $c \in (a, b)$ such that $h(c) = 0$, and thus $f(c) = g(c)$.
        \end{proof}
\end{enumerate}
\pagebreak

\begin{theorem}[Dirichlet's Approximation Theorem]
  For any real $\alpha$ and $N$, there exist integers $p, q$ such that
  \[ \abs{p\alpha - q} < \frac{1}{N}, \]
  with $1 \le q \le N$. By having a large enough $N$ we can make the difference as small as needed.
\end{theorem}

\begin{proof}[Lemma 1 Proof]
  Let $x \in \R, \eps > 0$. Suppose for the sake of contradiction there is no $n \in \N$ such that
  \[ \sin(n) \in N_{\eps}(\sin(x)) \setminus \set{\sin(x)}. \]
  So, our aim is to show that $n + 2\pi m$ for some $m \in \Z, n \in \N$ is within $\eps$ of $x$.
  By Dirichlet, we can make $\abs{v} = \abs{n + 2\pi m} < \eps$, and negating if necessary so that $n$ is positive.
  Then, multiply by $d \in \N$ such that $d(n + 2\pi m) \in N_{\eps}(x)$. Because $v$ is less than $\eps$, there is some multiple
  that lands it within $\eps$ of $x$ -\ think of splitting $\R$ into (infinitely many) $\eps$-sized subintervals:
  one of these clearly must contain $x$ and so a multiple of our approximation of $2\pi$ would necessarily also be within this interval.
  Thus, we say
  \[ \abs{dn + 2\pi m d - x} < \eps \implies \abs{\sin(dn + 2\pi m d) - \sin(x)} = \abs{\sin(dn) - \sin(x)} < \eps. \]
  Hence, $\sin(\N)$ is dense.
\end{proof}

\end{document}
