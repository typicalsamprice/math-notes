% -*- compile-command: "latexmk -pdf limits-cont-1.tex" -*-
\documentclass{article}

\input{template}

\author{Sam Price}
\date{}
\title{Limits and Continuity I}

\usepackage{pgfplots}

\begin{document}

\maketitle

\begin{enumerate}

  \item Consider the function $f\from \R \to \R$ defined as
        \[
        f(x) = \begin{cases}
          2x + 1 & x \ge 0\\
          -2x - 1 & x < 0.
        \end{cases}
        \]

        This is a walkthrough (by our Dear Dr. Spence, not I) for the proof that $\lim\limits_{x \to c} f(x) = f(c)$ for $c \ne 0$.

        \begin{enumerate}

          \item Clearly, $f$ is piecewise. As such, our proof should have 2 cases depending on $c \in \R$. Write the two cases down.

                The case of $c > 0$, in which case $f(c) = 2c+1$.

                The case of $c < 0$, when $f(c) = -2c - 1$.

          \item The Scratch Work Phase. Compute (and simplify) the expressions for $\abs*{f(x) - f(c)}$ for $x, c$ in your two cases above.

                If $x,c > 0$, we have
                \[ \abs*{f(x) - f(c)} = \abs*{2x + 1 - (2c + 1)} = 2\abs*{x - c}. \]
                Similarly, for $x, c < 0$ there is
                \[ \abs*{- 2x - 1 - (-2c - 1)} = 2\abs*{c - x} = 2\abs*{x - c}. \]

          \item Explain why it is important to assume \emph{both} $x$ and $c$ are in the same case for each expression.

                If $x < 0$ and $c > 0$ (or vice versa), it wouldn't be helpful as they would evaluate on totally different branches of $f$.
                This will be important later, because we are only choosing $x$ ``close enough'' to $c$.

          \item Now, let $\eps > 0$ and choose a $\delta > 0$ small enough so that $x, c$ are in the same branch (as I said!)
                and that $\abs{f(x) - f(c)} < \eps$. We \emph{can} choose a different $\delta$ for each case, but I am choosing one for both
                simultaneously, since that is possible for this specific $f$.

                Let $\eps > 0$. Since both $c, \eps$ are real numbers, there exists integers $m,n > 0$  such that we have
                \[ \frac{1}{m} < \abs{c}, \frac{1}{n} < \frac{\eps}{2}. \]
                Now, pick $\delta$ such that
                \[
                \delta = \min\set*{\frac{1}{m}, \frac{1}{n}} < \min\set*{\abs{c}, \frac{\eps}{2}}.
                \]

                Therefore, we find $\abs{x - c} < \delta \implies \abs{f(x) - f(c)} = 2\abs{x - c} < 2\delta < \eps$.

                The rationale behind taking the \emph{minimum} of the two values, is that if $\abs{c} < \eps/2$,
                we have to take $\delta < \abs{c}$, otherwise we may end up splitting the branches. This would be bad,
                since it's what we been trying to avoid this whole time.

                I only notice this while writing the next part: we can simply take the minimum, what I am doing trying to
                pick naturals with AP?\ Kept in for posterity.

          \item Write the proof that $\lim\limits_{x \to c}f(x) = f(c)$ if $c \ne 0$.

                \begin{proof}
                  Let $c \in \R \setminus \set*{0}, \eps > 0$. Pick $\delta$ by
                  \[ \delta = \frac{\min\set*{\abs{c}, \eps}}{2}. \]
                  Now, let $x \in \R$ such that $\abs{x - c} < \delta$.
                  Since $\abs{x - c} < \frac{\abs{c}}{2}$, we know that
                  \[ c - \frac{\abs{c}}{2} < x < c + \frac{\abs{c}}{2}. \]
                  As such, $c < 0 \implies x < 0$ and $c > 0 \implies x > 0$.

                  Consider now $\abs{f(x) - f(c)}$, and we notice that because $x$ and $c$ are in the same branch of $f$:
                  \begin{align*}
                    \abs{f(x) - f(c)} &= 2\abs{x - c}\\
                    &< 2\delta\\
                    &\le 2\parens*{\frac{\eps}{2}} = \eps.
                  \end{align*}

                  Therefore, $\lim\limits_{x \to c}f(x) = f(c)$ for $c \ne 0$.
                \end{proof}

          \item Now show that $\lim\limits_{x \to 0}f(x)$ does not exist.

                \begin{proof}[By negation of limit definition]
                  Fix $\eps = 1$, and let $\delta > 0$. We take
                  \[ x_{0} = -\min\set*{\frac{1}{2}, \frac{\delta}{2}} \implies \abs{x_{0}} < \delta.  \]
                  We also note that $x_{0} \in \lbrack -1/2 , 0 \rparen $.
                  Looking at the difference as values of $f$, we see
                  \begin{align*}
                    \abs{f(x_{0}) - f(0)} &= \abs{ - 2x_{0} - 1 - 1}\\
                                          &= 2\abs{-x_{0} - 1}\\
                                          &= 2\abs{1 + x_{0}} = 2(1 + x_{0}) \qquad \textrm{We are so strong, we bent the bars! Or, $1 + x_{0} > 0$.} \\
                                          &\ge 2 \cdot \frac{1}{2} = 1 = \eps.
                  \end{align*}
                  Therefore, the limit of $f$ at $0$ does not exist.
                \end{proof}

          \item Sketch the thing! Something about an $\eps$ and a $\delta$-neighborhood. Who knows??!

                Note that this is $\eps = 1, \delta = 0.65$, but we know $x_{0}$ is bounded below by $-0.5$ always.

                \begin{tikzpicture}[>=stealth]

                  \begin{axis}[
                    xmin=-1, xmax=1,
                    ymin=-2, ymax=4,
                    axis x line = middle,
                    axis y line = middle,
                    axis line style = <->,
                    xlabel = {$x$}, ylabel = {$y$},
                    xtick=\empty, ytick=\empty,
                    extra x ticks = {-0.325},
                    extra x tick labels = {}
                    ]
                    \addplot[no marks, black, ->] expression[domain=0:1, samples = 2]{2*x+1};
                    \addplot[no marks, black, <-] expression[domain=-1:0, samples = 2]{-2*x-1};

                    % Deco: https://tex.stackexchange.com/questions/246237/pgfplots-add-curly-rotated-braces-to-a-graph
                    % Epsilons
                    \draw [decorate, decoration={brace,amplitude=5pt,mirror}] (axis cs:0, 0) -- (axis cs:0,1)
                      node[midway, auto, swap, outer sep = 3pt, font = \tiny]{$\eps$};
                    \draw [decorate, decoration={brace,amplitude=5pt,mirror}] (axis cs:0, 1) -- (axis cs:0,2)
                      node[midway, auto, swap, outer sep = 3pt, font = \tiny]{$\eps$};
                    % Deltas
                    \draw [decorate, decoration={brace,amplitude=5pt,mirror}] (axis cs:-0.65, 0) -- (axis cs:0,0)
                      node[midway, xshift = 1mm, yshift = 0.5mm, auto, swap, outer sep = 3pt, font = \tiny]{$\delta$};
                    \draw [decorate, decoration={brace,amplitude=5pt,mirror}] (axis cs:0, 0) -- (axis cs:0.65,0)
                      node[midway, auto, swap, outer sep = 3pt, font = \tiny]{$\delta$};
                    % x_0
                    \node[anchor=south] at (axis cs:-0.325,0) {\tiny $x_{0}$};
                  \end{axis}

                \end{tikzpicture}

        \end{enumerate}


\end{enumerate}

\end{document}
