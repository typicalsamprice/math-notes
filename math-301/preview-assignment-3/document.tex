\documentclass{article}

\input{template}

\author{Sam Price}
\date{}
\title{Preview Assignment 3}

\begin{document}

\maketitle

\begin{enumerate}

  \item Let $a > 0$ and $A = \set{x \in \R : x^{2} < a}$. Explain why $A$ is bounded and nonempty, citing the textbook.\\
        Since $a > x^{2} \liff \sqrt{a} > \abs{x}$, we see by Definition 1.17 that $A$ is bounded above.
        By Definition 1.37 of an open interval, we see $A = \OpenInt{-\sqrt{a}, \sqrt{a}}$ is nonempty,
        since $-\sqrt{a} \ne \sqrt{a}$ for $a > 0$.

  \item \begin{enumerate}
          \item Since $\alpha^{2} > a$, and assuming there exists some $\eps > 0$ such that $\parens{\alpha - \eps}^{2} > a$, we see that
                \begin{equation*}
                  \alpha \ge b \implies \alpha^{2} \ge b^{2}.
                \end{equation*}
                Since $\alpha = \sup(A)$, we also know for any $\nu > 0$, $\alpha - \nu < b$ for some $b \in A$ (lots of strange variables here).
                This implies that $\parens{\alpha - \nu}^{2} < b^{2}$.
                Creating one large compound inequality, we see that
                \begin{equation*}
                  \parens{\alpha - \eps}^{2} < b^{2} \le \alpha^{2} < a < \parens{\alpha - \eps}^{2}.
                \end{equation*}
                However, this boils down to $\parens{\alpha - \eps}^{2} < \parens{\alpha - \eps}^{2}$, a clear contradiction.

          \item Now, if $\alpha^{2} < a$ and there is some $\eps > 0$ such that $\parens{\alpha + \eps}^{2} < a$,
                we see that $\alpha \ge b$, and that $\alpha - \eps < b$. Creating this inequality,
                \begin{equation*}
                  b^{2} \le \alpha^{2} < \parens{\alpha + \eps}^{2} < a.
                \end{equation*}

                However, if we take $c$ such that
                \begin{equation*}
                  c = \frac{\parens{\alpha + \eps}^{2} + a}{2},
                \end{equation*}
                then $\parens{\alpha + \eps}^{2} < c < a$, which implies that $\alpha < \sqrt{c}$.
                This however is a contradiction to $\alpha$ being the supremum, with $\sqrt{c} \in A$.

        \end{enumerate}

  \item The equality can be explained by simple binomial expansion, and the inequality holds since $n > 0$, and this
        means that $n^{-2} > 0$.

  \item We want to show there exists some $m \in \N$ such that
        \begin{equation*}
          \alpha^{2} - a > \frac{2\alpha}{m} > 0
        \end{equation*}
        Since $\alpha^{2} - a > 0$, we know there exists some $m \in \N$ by the Archimedean Property
        such that
        \begin{equation*}
          m(\alpha^{2} - a) > 2\alpha.
        \end{equation*}
        The claim follows naturally from this.

  \item Use the above to show
        \begin{equation*}
          \parens*{\alpha - \frac{1}{m}}^{2} - a > 0.
        \end{equation*}

        Note that from the previous part, we know there is some $m \in \N$ where
        \begin{equation*}
          \alpha - \frac{2\alpha}{m} > a.
        \end{equation*}
        It follows then, that
        \begin{equation*}
          \parens*{\alpha - \frac{1}{m}}^{2} > a + \frac{1}{m^{2}} > a.
        \end{equation*}

        This is sufficient to show $\alpha^{2} \le a$, since we can choose $\eps = 1/m$
        to reach the contradiction described in the first case.

  \item Let $\alpha^{2} < a$.
        Consider the value $a - \alpha^{2}$. Clearly, this value is positive, so we may find some
        $n \in \N$ where
        \begin{equation*}
          n\parens*{a - \alpha^{2}} > 2\alpha + 1.
        \end{equation*}

        Thus, we can see
        \begin{equation*}
          \alpha^{2} + \frac{1}{n}\parens*{2\alpha + 1} < a.
        \end{equation*}

        Continuing this, we find
        \begin{equation*}
          \alpha^{2}
          < \parens*{\alpha + \frac{1}{n}}^{2}
          \le \alpha^{2} + \frac{1}{n}\parens*{2\alpha + 1}
          < a.
        \end{equation*}

        Thus, when $\eps = 1/n$, we can reach our contradiction previously described for $\alpha^{2} < a$.

\end{enumerate}

Finally, we can say with certainty that $\alpha^{2} = a$. $\qedsymbol$

\end{document}
