% -*- compile-command: "latexmk -pdf document.tex" -*-
\documentclass{article}

\input{template}

\author{Sam Price}
\date{}
\title{Preview Assignment\\\Large{Derivatives I}}

\begin{document}

\maketitle

\begin{enumerate}


  \item Prove the quotient rule:

        % TeX SE: questions/25903
        \noindent\fbox{
          \parbox{\textwidth}{
            \textbf{Quotient Rule}: Let $f, g \from I \to \R$ be differentiable at $c \in I$ with $g(c) \ne 0$.
            \[ \parens*{\frac{f}{g}}'(c) = \frac{f'(c)g(c) - f(c)g'(c)}{g^{2}(c)} \]
          }
        }

        \begin{proof}
          Let $f, g \from I \to \R$ be differentiable at $c \in I$ and $g(c) \ne 0$.
          Therefore, we aim to find
          \[ \lim_{x \to c} \frac{f}{g}(x) =
            \lim_{x \to c} \frac{\frac{f(x)}{g(x)} - \frac{f(c)}{g(c)}}{x - c} \]
          We notice then
          \begin{align}
            \lim_{x \to c} \frac{\frac{f(x)}{g(x)}
            - \frac{f(c)}{g(c)}}{x - c} &= \lim_{x \to c}\frac{f(x)g(c) - f(c)g(x)}{(x - c)g(x)g(c)}\notag\\
            &= \lim_{x \to c} \frac{f(x)g(c) + [f(c)g(c) - f(c)g(c)] - f(c)g(x)}{x - c} \cdot \lim_{x \to c}\frac{1}{g(x)g(c)}\notag\\
            &= \lim_{x \to c} \frac{g(c)(f(x) - f(c)) - f(c)(g(x) - g(c))}{x - c} \cdot \frac{1}{{g(c)}^{2}}\\
            &= \lim_{x \to c} g(c)\frac{f(x) - f(c)}{x - c} - \lim_{x \to c} f(c)\frac{g(x) - g(c)}{x - c} \cdot \frac{1}{{g(c)}^{2}}\notag\\
            &= \frac{f'(c)g(c) - f(c)g'(c)}{{g(c)}^{2}}\notag.
          \end{align}

          Where we find (1) because $g$ is continuous (and nonzero already) at $c$.
        \end{proof}

  \item Define $g \from \R \to \R$ as
        \[ g(x) = \begin{cases}
          x^{3}\sin(1/x) & x \ne 0\\
          0 & x = 0.
        \end{cases} \]

        \begin{enumerate}
          \item Is $g$ differentiable at $x \ne 0$? Find a formula for $g'$ when $x \ne 0$.

                Yes, simply $g'(x) = 3x^{2}\sin(1/x) -x\cos(1/x)$, taking note that
                \[ \frac{d}{dx} \sin(1/x) = \cos(1/x) \cdot -\frac{1}{x^{2}}. \]

          \item Is $g$ differentiable at $x = 0$? Use the limit definition to check.

                \begin{align*}
                  g'(0) = \lim_{x \to 0}g(x) &= \lim_{x \to 0} \frac{g(x) - g(0)}{x - 0}\\
                  &= \lim_{x \to 0}\frac{x^{3}\sin(1/x)}{x}\\
                  &= \lim_{x \to 0}x^{2}\sin(1/x) = 0.
                \end{align*}
                Which is true as $x^{2}\sin(1/x)$ is continuous using Corollary 6.15 and
                the multiplication of $f, g \in C^{0}$ (and so the limit does equal its value at $x = 0$.)

          \item I genuinely don't know why you couldn't \textendash{} to me it looks like $g'$ \emph{is} continuous at 0?

        \end{enumerate}

  \item Let $f \from \R \to \R$. Show (a) that $f$ is differentiable on $\R$,
        and that (b) $f \notin D^{2}$. Define $f$ as
        \[ f(x) = \begin{cases}
          -x^{2} & x \le 0\\
          x^{2} & x > 0
        \end{cases} \]

        \begin{enumerate}

          \item Clearly, $f$ is differentiable away from 0 (the branches are differentiable),
                so let us only concern ourselves with $f'(0)$.

                Looking at this, we find
                \[
                  \lim_{x \to 0^{-}} \frac{f(x) - f(0)}{x} = \lim_{x \to 0^{-}}\frac{-x^{2}}{x} = \lim_{x \to 0^{-}} -x = -0 = 0,
                \]
                and for the right:
                \[
                  \lim_{x \to 0^{+}} \frac{f(x) - f(0)}{x} = \lim_{x \to 0^{+}}\frac{x^{2}}{x} = \lim_{x \to 0^{+}} x = 0.
                \]
                As such, $f'$ exists.

          \item Now, since $f$ is differentiable, we can find the formula where
                \[ f'(x) = \begin{cases}
                  - 2x & x \le 0\\
                  2x & x > 0.
                \end{cases} \]

                However, the issue is that when taking $f''$, $f'(x) = 2\abs{x}$ and so is not differentiable at $x = 0$ since
                the sign (of $x$) switches instantly. ``Formally'' put,
                \[ \lim_{x \to 0}\frac{2\abs{x}}{x} = ? \]
                is incomprehensible because $\pm 2$ isn't allowed (limits exist iff they are unique.)

        \end{enumerate}

\end{enumerate}

\end{document}
