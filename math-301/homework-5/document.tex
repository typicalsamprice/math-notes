% -*- compile-command: "latexmk -pdf document.tex" -*-
\documentclass{article}

\input{template}

\author{Sam Price}
\date{}
\title{Homework 5}

\begin{document}

\maketitle

\begin{enumerate}

  \item 7.3: Let $f \from \R \to \R = \floor{x}$, and let $C$ be the points at which $f$ is differentiable.
        Determine $C$ and $f' \from C \to \R$.

        Clearly, $f$  is discontinuous at integers, and so we can discount those from $C$ immediately. Then, let $c \in (n, n + 1)$ for some $n \in \Z$.
        Let $\eps > 0$, and pick $\delta < \min{\abs{c - n}, \abs{c - n - 1}}$. Then, $\abs{x - c} < \delta \implies \abs{f(x) - f(c)} = 0 < \eps$.

        Hence, $f'\from C \to \R = 0$ as well, since it is a flat line on each interval $(n, n + 1)$.

  \item Let
        \[ f_{a}(x) = \begin{cases} x^{a} & x \ge 0\\ 0 & x < 0 \end{cases} \]
        for some $a \in \R$.

        \begin{enumerate}
          \item For which $a$ is $f_{a}$ continuous at $x = 0$?

                Let $\eps > 0$.
                For $a > 0$, we see that picking $\delta = \eps^{1/a}$
                gives the result:
                \[ \abs{x} < \delta \implies \abs{f(x)} = \abs{x^{a}} = \abs{x}^{a} < \delta^{a} = \eps. \]

                For $a = 0$, $f(0) = 1$ and so is clearly discontinuous (for $\eps = 1/2$, any $\delta$ is insufficient by picking a ``small'' negative number.)

                For $a < 0$ we have a divide-by-zero quandry, since $x^{-a} = \frac{1}{x^{a}}$ and for $x = 0$, is $1/0$.

          \item For which $a$ is $f_{a}$ \emph{differentiable} at $x = 0$?

                Let us first look at the difference quotient:
                \[ \lim_{x \to 0}\frac{f(x) - f(0)}{x - 0} = \lim_{x \to 0}\frac{x^{a}}{x} = \lim_{x \to 0} x^{a-1}.\]
                For $a > 1$, this is clearly defined as simply 0.
                The reason $a = 1$ is differentiable is that
                \[ \lim_{x \to 0}\frac{f_{1}(x) - f_{1}(0)}{x - 0} = \lim_{x\to0}\frac{x}{x} = 1 \]
                which is still well-defined.

                For $a < 1$, we find that $b = a - 1 < 0$ and so $x^{b}$ (in the limit) is another division by zero -\ quite the no-no.

                Thus, $f_{a}$ is differentiable at 0 for $a \ge 1$.

        \end{enumerate}

\end{enumerate}

\end{document}
