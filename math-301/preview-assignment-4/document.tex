% -*- compile-command: "latexmk -pdf document.tex" -*-
\documentclass{article}

\input{template}

\author{Sam Price}
\date{}
\title{Preview Assignment 4\\\Large{Math 301}}

\begin{document}

\maketitle

\begin{enumerate}
  \item Prove that the sequence $(r^{n})$ converges for a fixed $r \in \R$ if and only if $r \in \lparen-1, 1\rbrack$, with $(r^{n}) \to 1$
        when $r = 1$ and $(r^{n}) \to 0$ when $r \in (-1, 1)$.\\
        \begin{proof}[$(r^{n})$ converges if $r \in \lparen -1, 1 \rbrack$]
          Assume by way of contradiction there exists some $r \notin \lparen -1, 1 \rbrack$ where $(r^{n})$ converges.
          Consider first $r = -1$, where
          \[
            (r^{n}) = (1, -1, 1, -1, 1, \ldots).
          \]
          This is not convergent, since it is clearly alternating. Now, consider the case $\abs{r} > 1$.
          With this, we see for any $n \in \N$
          \[
            \abs{r^{n}} = \abs{r}^{n} < \abs{r}^{n + 1} = \abs{r^{n + 1}}.
          \]
          Thus, the absolute value of each element in the sequence is unbounded, and cannot be convergent.
        \end{proof}
        \begin{proof}[$(r^{n})$ converges to 0 or 1]
          Consider the case $r = 1$. Then,
          \[
            (r^{n}) = (1, 1, 1, \ldots).
          \]
          Clearly, this is a constant sequence and therefore converges to 1.
          Similarly, the case $r = 0$ is the constant sequence $(0, 0, \ldots) \to 0$.

          Now, consider $0 < \abs{r} < 1$ and let $\eps > 0$.
          Choose $N$ such that
          \[
            N > \frac{\ln \eps}{\ln \abs*{r}}.
          \]
          This would then mean, (since $\ln \abs*{r} < 0$) that
          \begin{align*}
            N &> \frac{\ln \eps}{\ln \abs*{r}}\\
            \implies N \ln \abs*{r} &< \ln \eps\\
            \ln \abs*{r}^{N} &< \ln \eps\\
            \abs{r}^{N} = \abs{r^{N}} &< \eps.
          \end{align*}
          Thus, for any $n > N$, we would find
          \begin{align*}
            \abs*{r^{n} - 0} = \abs{r^{n}} < \abs{r^{N}} < \eps.
          \end{align*}
          Therefore, $(r^{n})$ converges to 0 when $r \in (-1, 1)$, and in general $(r^{n})$ converges when $r \in \lparen -1, 1 \rbrack$.

        \end{proof}

  \item The Harmonic divergence proof.
    \begin{enumerate}
      \item The proper negation is\\
            A sequence $(a_{n})$ is \emph{not} Cauchy if $\exists \epsilon > 0$ where $\forall N \in \N$
            there exist $m, n > N$ such that
            \[
            \abs*{a_{n} - a_{m}} \ge \eps.
            \]

      \item Find an expression for the difference of the partial sums $s_{k} - s_{\ell}$ for $k > \ell$.\\
            Let $k > \ell \ge 1$. Then, the difference between the partial sums is
            \begin{align*}
            s_{k} - s_{\ell} &= \frac{1}{1} + \frac{1}{2} + \frac{1}{3} + \frac{1}{4} + \cdots + \frac{1}{\ell} + \cdots + \frac{1}{k}\\
              &- \parens*{\frac{1}{1} + \frac{1}{2} + \frac{1}{3} + \frac{1}{4} + \cdots + \frac{1}{\ell}}\\
              &= \frac{1}{\ell + 1} + \frac{1}{\ell + 2} + \cdots + \frac{1}{k}\\
            \end{align*}

      \item\label{ineq}
            Using your expression above for $k > \ell$, show that $s_{k} - s_{\ell} > 1 - \frac{\ell}{k}$.\\
            Note that we can find a sum that is less than $s_{k} - s_{\ell}$, namely:
            \begin{align*}
              s_{k} - s_{\ell} &= \underbrace{\frac{1}{\ell + 1} + \frac{1}{\ell + 2} + \cdots + \frac{1}{\ell + (k - \ell)}}_{k - \ell \textrm{ terms }}\\
              &\ge \underbrace{\frac{1}{k} + \frac{1}{k} + \cdots + \frac{1}{k}}_{k - \ell \textrm{ terms }}\\
              &= \frac{k - \ell}{k} = 1 - \frac{\ell}{k}.
            \end{align*}

            With equality only when $k = \ell + 1$. While this isn't the strict inequality we hoped for, it is the best we can do
            since the \emph{actual value} of $s_{\ell + 1} - s_{\ell}$ is specifically equal to $1/(\ell + 1)$.

      \item Show that there is a choice so that $s_{k} - s_{\ell} > \frac{1}{2}$.\\
            Since we know from part~\ref{ineq} that
            \[
            s_{k} - s_{\ell} \ge 1 - \frac{\ell}{k},
            \]
            we can choose $k = 2\ell > \ell + 1$ so that
            \[
            s_{2\ell} - s_{\ell} > 1 - \frac{\ell}{2\ell} = 1 - \frac{1}{2} = \frac{1}{2}.
            \]

      \item Using this work, prove
            \[
            \sum_{n = 1}^{\infty}\frac{1}{n}
            \]
            diverges. Use Theorem 3.42 (Cauchy iff convergent.)\\
            \begin{proof}
              Assume by way of contradiction that the harmonic sum converges, and is therefore Cauchy (and vice versa).
              Let $s_{k}$ be the $k$\textsuperscript{th} partial sum, and since the harmonic series converges so must $(s_{k})$.
              As such, $(s_{k})$ is Cauchy, and so for each $\eps > 0$ there exists some $N \in \N$ such that for all $n > m > N$
              we have
              \[
                \abs*{s_{n} - s_{m}} < \eps.
              \]
              Using our inequalities from part 2c and 2d, we see that we may choose an $n \ge 2m$ such that for $0 < \eps < \frac{1}{2}$
              \[
                \abs*{s_{n} - s_{m}} = s_{n} - s_{m} > \frac{1}{2} > \eps.
              \]
              However, this is a contradiction to the Cauchy-ness of $(s_{k})$, since not all $\eps > 0$ would be able to ``contain''
              the difference for \emph{all} $n, m > N$. This means that $(s_{k})$ is not Cauchy, and by Theorem 3.42 not convergent.
              Thus, the harmonic series diverges as well.
            \end{proof}
    \end{enumerate}

\end{enumerate}


\end{document}
