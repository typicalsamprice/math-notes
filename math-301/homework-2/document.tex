% -*- compile-command: "latexmk -pdf document.tex" -*-
\documentclass{article}

\input{template}

\author{Sam Price}
\title{Homework 2\\\Large{Math 301}\\\large{Part 2: 4\textendash{}10}}

\begin{document}

\maketitle

% Exs. start page 120 of PDF

\begin{enumerate}
%  \item Prove the following sequences converge to the given value.
        \begin{enumerate}
          \item Let $a_{n} = 7 - \frac{1}{\sqrt{n}}$. Show that $(a_{n})$ converges to 7.
                \begin{proof}
                  Let $\eps > 0$, and note that $\eps^{2} > 0$ also. We may choose by the Archimedean Property some $N \in \N$
                  large enough such that
                  \[
                    \frac{1}{N} < \eps^{2}.
                  \]
                  Thus, for all $n > N$ we find
                  \[
                    \frac{1}{n} < \frac{1}{N} < \eps^{2}
                  \]
                  and therefore
                  \[
                    \frac{1}{\sqrt{n}} < \frac{1}{\sqrt{N}} < \eps.
                  \]

                  Thus, we see that for all $n > N$,
                  \[
                    \abs*{a_{n} - 7} = \abs*{7 - \frac{1}{\sqrt{n}} - 7} = \abs*{-\frac{1}{\sqrt{n}}} = \frac{1}{\sqrt{n}} < \eps
                  \]
                \end{proof}


          \item Let $a_{n} = \dfrac{2n - 2}{5n + 1}$. Show that $(a_{n}) \to \dfrac{2}{5}$.
                \begin{proof}
                  Let $\eps > 0$. Note that
                  \begin{align*}
                    \abs*{\frac{2n-2}{5n+1} - \frac{2}{5}} &= \abs*{\frac{5(2n-2) - 2(5n + 1)}{5(5n + 1)}} = \abs*{\frac{-12}{25n + 5}}\\
                    &< \abs*{\frac{-12}{25n}} = \frac{12}{25n}.
                  \end{align*}

                  By choosing $N$ (by the Archimedean Property) such that
                  \[
                    0 < N\inv < \eps \cdot \frac{25}{12}
                  \]
                  we find that for all $n > N$,
                  \[
                    \frac{12}{25}\cdot \frac{1}{n} < \frac{12}{25} \cdot N\inv < \frac{12}{25}\parens*{\eps \cdot \frac{25}{12}}.
                  \]

                  Adding this inequality to the inequality described before, we see that when $n > N$
                  \[
                    \abs*{a_{n} - \frac{2}{5}} < \frac{12}{25n} < \frac{12}{25N} < \eps.
                  \]
                \end{proof}

          \item Let $a_{n} = 7 - \dfrac{1}{\sqrt{n + \sqrt{n} + 13}}$. Prove $(a_{n}) \to 7$. ``By the monotone convergence theorem.''
                \begin{proof}
                  Let $\eps > 0$, and note that $\eps^{2} > 0$. We see that
                  \[
                    \abs*{a_{n} - 7} = \abs*{\frac{1}{\sqrt{n + \sqrt{n} + 13}}} = \frac{1}{\sqrt{n + \sqrt{n} + 13}}.
                  \]
                  Now, choose $N$ such that
                  \[
                    \frac{1}{N} < \eps^{2}.
                  \]

                  We see that for $n > N$, clearly
                  \[
                    \frac{1}{\sqrt{n + \sqrt{n} + 13}} < \frac{1}{\sqrt{n}} < \frac{1}{\sqrt{N}} < \eps.
                  \]

                  Finalizing our inequality chain, we determine for all $n > N$
                  \[
                    \abs*{a_{n} - 7} = \frac{1}{\sqrt{n + \sqrt{n} + 13}} < \frac{1}{\sqrt{n}} < \eps.
                  \]
                \end{proof}
        \end{enumerate}

%\setcounter{enumi}{1} % We are setting this here lol
\item Let $(a_{n}) \subset \R$ converge to $a \in \R$.
    Given any real number $k$, show that $(k + a_{n}) \to k + a$ and $(ka_{n}) \to ka$.

    \begin{proof}
        Let $\eps > 0$. We wish to show that
        \[
        \abs*{k + a_{n} - (k + a)} < \eps.
        \]

        By looking at the left side, we see that
        \begin{align*}
        \abs*{k + a_{n} - (k + a)} = \abs*{a_{n} - a}.
        \end{align*}

        Since $(a_{n}) \to a$, we know this is always less than $\eps$ for all $n > N$, for some $N$.
        Therefore, we can say that for all $n > N$, as appropriate for $(a_{n})$,
        \[
        \abs*{k + a_{n} - (k + a)} < \eps \implies (k + a_{n}) \to k + a.
        \]
    \end{proof}

    For the multiplicative sequence:
    \begin{proof}
        Let $\eps > 0$. Note that
        \[
        \abs*{ka_{n} - ka} = \abs*{k(a_{n} - a)} = \abs*{k}\abs*{a_{n} - a}.
        \]

        Since $(a_{n}) \to a$, we know that given any $\eps > 0$ there is some point $N$
        where for all $n > N$, $\abs*{a_{n} - a} < \eps$. Assume that $k \ne 0$, since it should be clear
        that a sequence of all zeroes converges to 0.

        Now, consider the inequality
        \[
        \abs*{a_{n} - a} < \dfrac{\eps}{\abs*{k}}.
        \]

        Since $\abs{k}$ > 0, $\eps/\abs{k}$ is as well. Thus, there is some $N \in \N$ where for all $n > N$
        the above is true, and in fact
        \[
        \abs*{ka_{n} - ka} < \eps \implies (ka_{n}) \to ka.
        \]
    \end{proof}

%  \item Assume $(a_{n})$ is bounded and $(b_{n}) \to 0$. Show that $(a_{n}b_{n}) \to 0$.
        \begin{proof}
          Since $(a_{n})$ is bounded, $\abs{a_{n}} < C$ for all $n$ (where $C$ is not minimal, but \textit{is} finite).
          The reason $<$ is the proper relation here instead of $\le$ is to avoid dividing by zero.
          Then, we see
          \[
            \abs*{a_{n}b_{n}} = \abs{a_{n}}\abs{b_{n}} \le C\abs*{b_{n}}.
          \]

          Let $\eps > 0$, and note that $\eps/C > 0$ also. Choose $N$ so that $\abs{b_{n}} < \eps/C$ for all $n > N$.
          Then for $n > N$,
          \[
            \abs*{a_{n}b_{n} - 0} < \abs*{Cb_{n}} = C\abs*{b_{n}} < C \cdot \frac{\eps}{C} = \eps
          \]

          Thus, $(a_{n} \cdot b_{n}) \to 0$ as well.
        \end{proof}


\setcounter{enumi}{3} % We are setting this here lol
\item Let $a_{1} = 1$, and define the sequence $(a_{n})$ such that
\[
  a_{n + 1} = \sqrt{2 + a_{n}}.
\]

Determine whether $(a_{n})$ converges, and if so the limit.

\begin{proof}
  Firstly, we want to show that $(a_{n})$ is bounded.
  We see that $a_{2} = \sqrt{3} < 2$. Now, assume $a_{n} < 2$ for some $n$, and consider $a_{n + 1}$.
  \[
    a_{n + 1} = \sqrt{2 + a_{n}} < \sqrt{2 + 2} = 2.
  \]

  Thus, we know $(a_{n})$ is bounded by 2. Now, let us prove $(a_{n})$ is monotonic.
  For the case $n = 1$, we see
  \[
    a_{n + 1} - a_{n} = \sqrt{3} - 1 \approx 1.73 > 0.
  \]
  Now, assuming this holds for some $n > 0$, we note
  \begin{align*}
    a_{n + 2} - a_{n + 1} &= \sqrt{a_{n + 1} + 2} - \sqrt{a_{n} + 2}\\
    &= \frac{a_{n + 1} + 2 - a_{n} - 2}
      {\sqrt{a_{n + 1} + 2} + \sqrt{a_{n} + 2}}\\
    &= \frac{a_{n + 1} - a_{n}}{\sqrt{a_{n + 1} + 2} + \sqrt{a_{n} + 2}} > 0.
  \end{align*}
  Since we know the numerator \emph{and} denominator are positive numbers, so the difference $a_{n + 2} - a_{n + 1}$ is as well.
  Thus, $(a_{n})$ is monotone increasing. By MCT, $(a_{n})$ then converges.
  Now, if $L = \lim_{n \to \infty} a_{n} = \lim_{n \to \infty}a_{n + 1} \implies L = \sqrt{2 + L}$.
  Thus, solving for $L$, we see
  \begin{align*}
    L &= \sqrt{L + 2}\\
    L^{2} -L - 2 &= 0\\
    (L - 2)(L + 1) &= 0.
  \end{align*}
  Thus, we then know $L = -1$ or $2$. We know the limit is positive though, and as such we can say $(a_{n}) \to 2$.
\end{proof}


  \item Show, directly from the definition, that the sequence given by
        \[
        a_{n} = \frac{n + 1}{n} = 1 + \frac{1}{n}
        \]
        is Cauchy.\\

        \begin{proof}
           Let $\eps > 0$. Choose $N$ such that
           \[
           \frac{1}{N} < \eps.
           \]
           Thus, for all $m > n > N$, we find that
           \begin{align*}
           \abs*{a_{n} - a_{m}} &= \abs*{1 + \frac{1}{n} - \parens*{1 + \frac{1}{m}}}\\
           &= \abs*{\frac{1}{n} - \frac{1}{m}}\\ &< \frac{1}{n} < \frac{1}{N} < \eps.
           \end{align*}
           Hence, $(a_{n})$ is Cauchy.
        \end{proof}

  \item Suppose $(a_{n}) \to a$. Show that the sequence $(b_{n})$ defined by
        \[
        b_{n} = \frac{\sum_{i = 1}^{n}a_{i}}{n}
        \]
        also converges to $a$.

        The content of this proof has been removed in accordance with our DMCA policy (see note on the Ces\`aro summation proof.)
        This problem will be submitted alone at a later date possibly. Was not in the mood to rework it at 2:02am.

  \item \begin{enumerate}
          \item\label{alt-1} Converges by alternating series test, but is not absolutely convergent since $1/\sqrt{k} \ge 1/k$ and the Harmonic Series diverges.
          \item Diverges. Terms approach $\pm 1$ which is not zeor.
          \item Absolutely Convergent, geometric sequence where $r = \abs*{(\ln 4)\inv} < 1$
          \item This is just $e$, and so converges (absolutely).
          \item This sum can be continuously rewritten such that the $k$th partial sum is equal to $\sqrt{k + 2} - 1$.
                Since that is unbounded, it must diverge. The indexing is messed up but the errata mentions it so I won't say anything else about that.
          \item Diverges, every term is a constant greater than zero.
          \item Converges, since each term goes to 0 while alternating much like part (a).
          \item Looking at the function corresponding to each term, we see (noting that $\ln \ln 53 \approx 3.7$)
                \[
                \odf{x}\parens*{\frac{\ln x}{\ln \ln x}} = \frac{\ln(\ln(x)) - 1}{x \ln^{2}(\ln(x))} > 0 \quad \forall x \ge 53.
                \]
                Since each term is only growing, it must diverge.
          \item This must diverge since for $n \ge 7$ each term is greater than the same term from the Harmonic series, meaning the ``sum''
                is greater than it. Since it diverges, so must this series.
        \end{enumerate}

  \item Given a series $\displaystyle \sum_{k = 1}^{\infty}a_{k}$ with $a_{k} \ne 0$, assume that
        \[ r := \lim_{k \to \infty} \abs*{\frac{a_{k + 1}}{a_{k}}} < 1. \]
        \begin{enumerate}
          \item Let $q \in \R$ such that $r < q < 1$. Explain why there is some $N$ such that $n \ge N$ implies
                that $\abs{a_{n + 1}} \le \abs{a_{n}} \cdot q$.

                Let $\eps > 0$, and note that if we have the above limit defining $r$, there is the sequence
                \[ \frac{\abs{a_{n + 1}}}{\abs{a_{n}}} \]
                that must converge to $r$. Therefore, we can say there exists an $N$ such that
                \[ \abs*{\frac{\abs{a_{n + 1}}}{\abs{a_{n}}} - r} < \eps \ \parens*{\le \eps} \quad \forall n \ge N. \]
                Then, deconstructing the absolute value, we find
                \[
                r -\eps < \frac{\abs{a_{n + 1}}}{\abs{a_{n}}} \le r + \eps.
                \]
                Choosing $\eps = q - r > 0$, we will then uncover for $n \ge N$ (for an appropriate such $N$),
                \[
                \frac{\abs{a_{n + 1}}}{\abs{a_{n}}} \le r + (q - r) = q \implies \abs{a_{n + 1}} \le \abs{a_{n}} \cdot q.
                \]
                Further, for $k \in \N \union \set{0}$ and $n \ge N$ we see that
                \begin{equation}\label{eq:pow-qk} \abs{a_{n + k}} \le \abs{a_{n}} \cdot q^{k}. \end{equation}

          \item Explain why $\sum_{k = 1}^{\infty}\abs{a_{N}} \cdot q^{k}$ must converge.\\
                This is a geometric sequence with a common multiple of $q$. Namely,
                \[ \sum_{n = 1}^{\infty}\abs{a_{N}}q^{n} = \frac{\abs{a_{N}}}{1 - q} - \abs{a_{N}} = \frac{\abs{a_{N}}\cdot q}{1 - q}. \]
          \item Use part (b) to prove that $\sum\abs{a_{k}}$ converges.

                Note that we can see given our $N$ from part (a)
                \[
                \sum_{n = 1}^{\infty}\abs{a_{n}} = \parens*{\sum_{n = 1}^{N - 1}\abs{a_{n}}} + \parens*{\sum_{n = N}^{\infty}\abs{a_{n}}}.
                \]
                Using our $q \in (r, 1)$ and geometric series from part (b), we can find
                \begin{align*}
                  \sum_{n = 1}^{\infty}\abs{a_{n}} &= \parens*{\sum_{n = 1}^{N - 1}\abs{a_{n}}} + \parens*{\sum_{n = N}^{\infty}\abs{a_{n}}}\\
                  &\le \parens*{\sum_{n = 1}^{N - 1}\abs{a_{n}}} + \parens*{\sum_{n = N}^{\infty}\abs{a_{N}}q^{n}} \quad \textrm{ by~\eqref{eq:pow-qk} }\\
                  &\le \parens*{\sum_{n = 1}^{N - 1}\abs{a_{n}}} + \frac{\abs{a_{N}} \cdot q}{1 - q}.\\
                \end{align*}
                This last sum is clearly a finite number, say $V$, and thus $\sum\abs{a_{n}} \le V$ implies that the series itself
                is convergent (and the original series $\sum a_{k}$ $\therefore$ being absolutely convergent).
        \end{enumerate}

  \item A series $\displaystyle\sum_{n = 1}^{\infty}a_{n}$ is said to be \emph{Ces\`aro summable} if the limit
        \[ \lim_{n \to \infty} \frac{s_{1} + \cdots + s_{n}}{n} \]
        exists, with $s_{n}$ denoting the $n$\textsuperscript{th} partial sum.
        \begin{enumerate}
          \item Prove that convergence implies Ces\`aro summability.

                I realize now, in the early morning hours of March 4\textsuperscript{th} after all of this is done,
                that problem 6 is literally this exact idea basically. Even moreso, that means I did number six wrong, since it would only work
                if the entire sequence in that scenario was positive.
                \begin{proof}
                  Let $\eps > 0$ and $(s_{n}) \to s$ be the partial sums of a convergent series.
                  Now, let $(c_{n})$ be the sequence of the averages of the first $n$ partial sums, defined by
                  \[ c_{n} = \frac{1}{n}\sum_{i = 1}^{n}s_{i}. \]
                  There is the sequence $(\delta_{n})$ as well which is
                  \[ \delta_{n} = s_{n} - s. \]
                  With the idea that $(\delta_{n}) \to 0$ (equivalent to $(s_{n}) \to s$, but will be easier for me to track).
                  Choose $N$ such that for $n > N$ we have
                  \[ \abs{\delta_{n}} < \frac{\eps}{3}. \]
                  Notice then that for $n > N$ there is
                  \begin{align*}
                    c_{n} &= \frac{s_{1} + \cdots + s_{N}}{n} + \frac{s_{N + 1} + \cdots + s_{n}}{n}\\
                    &= \frac{s_{1} + \cdots s_{N}}{n} + \frac{\delta_{N + 1} + \cdots + \delta_{n}}{n} + \frac{n - N}{n}s.
                  \end{align*}
                  Note the extra term that is a multiple of $s$ to balance out the extra subtracted terms within elements of $(\delta_{n})$.

                  Using this, we find that
                  \begin{align*}
                    \abs*{c_{n} - s}
                    &= \abs*{\frac{s_{1} + \cdots + s_{N}}{n} + \frac{\delta_{N + 1} + \cdots + \delta_{n}}{n}
                      + \frac{n - N}{n}s - s}\\
                    &\le \frac{\abs*{s_{1} + \cdots + s_{N}}}{n} + \frac{\abs{\delta_{N + 1} + \cdots + \delta_{n}}}{n}
                      + \frac{N}{n}\abs{s}\\
                    &\le \frac{\abs*{s_{1} + \cdots + s_{N}}}{n} + \frac{\abs{\delta_{N + 1}} + \cdots + \abs{\delta_{n}}}{n}
                      + \frac{N}{n}\abs{s}\\
                    &< \frac{\abs*{s_{1} + \cdots + s_{N}}}{n} + \frac{n - N}{n}\cdot\frac{\eps}{3}
                      + \frac{N}{n}\abs{s}\\
                  \end{align*}
                  Now, if we take $M > N$ large enough such that
                  \[ \frac{1}{M} < \frac{\eps}{3\abs*{s_{1} + \cdots + s_{N}}} \]
                  and
                  \[ \frac{1}{M} < \frac{\eps}{3N\abs{s}}, \]
                  we find for $n > M > N$ there is
                  \begin{align*}
                    \abs{c_{n} - s} &< \frac{\abs{s_{1} + \cdots + s_{N}}}{n} + \frac{n - N}{n}\cdot\frac{\eps}{3} + \frac{N}{n}\abs{s}\\
                    &< \frac{\eps}{3} + \frac{\eps}{3} + \frac{\eps}{3} = \eps.
                  \end{align*}
                  Thus, convergence implies Ces\`aro-summability.
                \end{proof}

          \item Give an example of a Ces\`aro-summable series that does not converge.

                The simplest answer which feels unfair is the infamous Grandi's series, which is
                \[ \sum_{n = 1}^{\infty}\parens*{-1}^{n - 1} = 1 - 1 + 1 - 1 + 1 - \cdots \]
                where the Ces\`aro sum is simply $\ds\frac{1}{2}$. This is because
                \[
                \lim_{n \to \infty}\frac{1 + 0 + 1 + 0 + \cdots}{n} = \lim_{n\to\infty}\frac{n/2}{n} = \frac{1}{2}.
                \]
        \end{enumerate}

  \item Prove that if an absolutely convergent series $\ds\sum a_{k} = a$ is unconditionally convergent to $a$.

        \begin{proof}
          Let the series $\ds\sum_{k = 1}^{\infty}a_{k} = a$, and $\ds\sum_{k = 1}^{\infty}\abs{a_{k}}$ is finite.
          Let $\eps > 0$, and since this series is absolutely convergent there exists an $N$ where for all $n \ge N$ we have both
          \begin{equation}\label{abs-conv-defn}
            \abs*{\sum_{k = 1}^{n}a_{k} - a} < \frac{\eps}{2} \quad \textrm{ and } \quad \sum_{k = n}^{\infty}\abs{a_{k}} < \frac{\eps}{2}.
          \end{equation}
          Note that the $N$ may not be minimal for either case, but taking the greater of the two minimal $N$s lends us to the
          above condition. Now, let $\pi$ be a permutation of $\N$ and $\ds\sum_{k = 1}^{\infty}a_{\pi(k)}$ be a rearrangement of the series.
          Choose some $M \ge N$ large enough such that
          \[ \set*{a_{1}, \ldots, a_{N}} \subseteq \set*{a_{\pi(1)}, \ldots, a_{\pi(M)}}. \]
          Note that this $M$ must be finite since $\pi$ cannot permute any natural number to something semantically
          suspicious such as ``second to last''.

          We then see for $n > M$
          \begin{equation}\label{ineq-permuted-to-M} \sum_{k = 1}^{N}a_{k} \le \sum_{k = 1}^{n}a_{\pi(k)}. \end{equation}
          Let $D = \pi\inv(\set{1, \ldots, N})$, and we see that in fact the extra indices (pre-permutation)
          included in the permuted sum to $M$ is
          \[ I = \set*{1, \ldots, n} \setminus D. \]
          Notice by the definition of $I$ and $D$ there are the equalities
          \begin{align}
            \sum_{i \in I}a_{\pi(i)} + \sum_{i \in D}a_{\pi(i)} = \sum_{k = 1}^{n}a_{\pi(k)}. \label{split-permuted-full-sum}\\
            \sum_{i \in D}a_{\pi(i)} = \sum_{k = 1}^{N}a_{k}. \label{sum-over-D-eq-up-to-N}
          \end{align}

          Now, let $S$ and $T$ be the minimum and maximum values of $\pi(I)$ respectively.
          Using these, we find
          \begin{align*}
            \abs*{\sum_{i \in I}a_{\pi(i)}} &\le \sum_{i \in I}\abs*{a_{\pi(i)}}\\
            &\le \sum_{k = S}^{T}\abs*{a_{k}}\quad \textrm{ because the permuted values of } \pi(I) \subseteq \set*{S, \ldots, T}.\\
            &\le \sum_{\mathclap{ k = N }}^{\infty}\abs*{a_{k}} \quad \textrm{ since } S > N \textrm{ and clearly } T < \infty.\\
            &< \frac{\eps}{2} \quad \textrm { by~\eqref{abs-conv-defn} }.
          \end{align*}
          Now, to show that $\sum_{k = 1}^{\infty}a_{\pi(k)} = a$:
          \begin{align*}
            \abs*{\sum_{k = 1}^{n}a_{\pi(k)} - a}
            &= \abs*{\sum_{i \in D}a_{\pi(i)} + \sum_{i \in I}a_{\pi(i)} - a} \quad \textrm{ with~\eqref{split-permuted-full-sum} }\\
            &\le \abs*{\sum_{i \in D}a_{\pi(i)} - a} + \abs*{\sum_{i \in I}a_{\pi(i)}}\\
            &< \abs*{\sum_{i \in D}a_{\pi(i)} - a} + \frac{\eps}{2}\\
            &=\abs*{\sum_{k = 1}^{N}a_{k} - a} + \frac{\eps}{2} \quad \textrm{ by~\eqref{sum-over-D-eq-up-to-N} }\\
            &< \frac{\eps}{2} + \frac{\eps}{2} \quad \textrm{ by~\eqref{abs-conv-defn} }\\
            &= \eps.
          \end{align*}
          Thus, absolutely convergent series are also unconditionally convergent in $\R$.
        \end{proof}
\end{enumerate}

\end{document}
