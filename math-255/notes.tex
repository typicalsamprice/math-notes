% -*- compile-command: "latexmk -pdf notes.tex" -*-
\documentclass{article}

\input{template}
%\usepackage{lmodern}

\title{Calculus \& Analytic Geometry III}
\author{Sam Price}

\begin{document}

\maketitle

\section{What is a vector?}

A vector, usually denoted $\vec{v}$ or $\vec{u}$, is an element of a ``vector space'', sometimes generically named $V$. However, the definition
``a vector is an element of a vector space'' is not particularly heplful. In this course, vectors will always be elements of $\R^{3}$ (or sometimes $\R^{2}$).
We may write a vector element-wise in ``bracketed notation'', or as a sum of ``basis vectors'' (either you understand from linear algebra, or you will someday.)
\[ \vec{v} = \veclit{x, y, z} = x\ihat + y\jhat + z\khat. \]

\subsection{Vector Operations}

As would be intuitive, addition and subtraction are performed component-wise
\[ \veclit{a, b, c} + \veclit{x, y, z} = \veclit{a + x, b + y, c + z}. \]
Similarly, ``scalar multiplication'' for any $a \in \R$ is intuitive as well, in that
\[ a\veclit{x, y, z} = \veclit{ax, ay, az}. \]

\section{Non-Cartesian Coordinate Systems}
\subsection{Cylindrical Coordinates}
In two dimensions, (cylindrically) polar coordinates are fairly simply mapped, in that
\[ (x, y) \mapsto (r, \theta) \]
with
\[ r = \sqrt{x^{2} + y^{2}} \quad\text{and}\quad \theta = \cos\inv\parens*{\frac{x}{r}} \]
and the exception being when $r = 0$, then $\theta = 0$ as well.

In three dimensions, we see the addition of a simple $z$-axis, and
\[ (x, y, z) \mapsto (r, \theta, z) \]
with the normal conversions, taking $z$ alone for the third element of the tuple.

\subsection{Spherical Coordinates}
These only exist in 3 dimensions, and have the form
\[ x = (\rho, \theta, \phi) \]
with $\rho$ being the distance from the origin, $\theta$ the angle respective to the positive $x$-axis
and $\phi$ being the angle with respect to the positive $z$-axis. To convert from normal polar, we would see
\[ (r, \theta) \mapsto (r, \theta, \pi/2). \]

\section{Change of Variables}
Polar, cylindrical and spherical coordinates are essentially $u$-substitutions in multiple integrals.

\underline{Two Rules:}
The differentials $dx, dy, dr, d\theta$, etc. are all examples of \emph{differential forms}.
For any differential forms $du, dv$ we have
\begin{itemize}
  \item $\dx{u}^{2} = 0$.
  \item $\dx{u}\, \dx{v} = - \dx{v}\dx{u}$.
\end{itemize}

\begin{ex}
  Consider $x = r\cos\theta$, and we find
  \[ \dx{x} = \cos\theta \dx{r} - \sin\theta \cdot  r\dx{\theta}. \]
  Similarly for $y$, we see
  \[ \dx{y} = \sin\theta \dx{r} + \cos\theta \cdot r\dx{\theta}. \]
  And then we see
  \[ \dx{x}\dx{y} = \cos^{2}\theta \cdot r\dx{r}\dx{\theta} - \sin^{2}\theta\cdot r\dx{\theta}\dx{r} = r(\cos^{2}\theta + \sin^{2}\theta)\dx{r}\dx{\theta} = r\dx{r}\dx{\theta}.  \]
  This is the reason behind the extra $r$ showing up in polar integrals.
\end{ex}

\subsection{General Transformations}
For any transformation $T$ that satisfies
\begin{itemize}
  \item $T \in C^{1}$.
  \item $T$ is bijective.
\end{itemize}
The differential form $\idx{A} = \dx{u}\dx{v}$ is changed such that the conversion from $(u, v)$ to $(x, y)$ looks like
\begin{align*}
  x &= x(u, v)\\
  y &= y(u, v)
\end{align*}
and by using vectors, we may combine them as
\[ \vec{r}(u, v) = \veclit{x(u, v), y(u, v)}. \]

This lends itself nicely to the approximation of the transformations for $\idx{u}$ and $\idx{v}$ by using the partials of $\vec{r}$.
Using the cross product, we find
\[ \dx{A} = \norm*{\grad \vec{r_{u}}\dx{u} \cross \vec{r_{v}}\dx{v}} = \norm{\vec{r_{u}} \cross \vec{r_{v}}}\dx{u}\dx{v}. \]
Simplifying, we see
\begin{align*}
  \dx{A} &= \parens*{\frac{\del x}{\del u} \cdot \frac{\del y}{\del v} - \frac{\del x}{\del v}\cdot\frac{\del y}{\del u}}\dx{u}\dx{v}\\
  &= \begin{vmatrix}[1.5]
    \frac{\del x}{\del u} & \frac{\del y}{\del u}\\
    \frac{\del x}{\del v} & \frac{\del y}{\del v}
  \end{vmatrix} \dx{u}\dx{v}.
\end{align*}

Which is (partially) the \emph{Jacobian} of $T$.

\begin{theorem}
  If $T \in C^{1}$, is bijective and has a nonzero Jacobian $\mathbf{J}$, then
  \[
    \iint_{R} f(x, y) \dx{x}\dx{y} = \iint_{S} f(x(u, v), y(u, v))\mathbf{J}\dx{u}\dx{v}.
  \]
\end{theorem}

\section*{Green's Theorem}
For a vector field $\vec{F} = \veclit{P, Q}$, we can say
\[ \int_{\del A} \vec{F} \cdot \dx\vec{r} = \iint_{A} \frac{\del Q}{\del x} - \frac{\del P}{\del y} \dx A.  \]

In fact, the integrand comes from the determinant:
\[ \left|\begin{matrix}
  \frac{\del}{\del x} & \frac{\del}{\del y}\\
  P & Q
\end{matrix}\right| = \grad \cross \vec{F} = \curl(\vec{F}). \]

\section*{16.5}
In 16.3, we found several results to show that a 2-component vector field $V = \veclit{P, Q}$ is conservative if:
\begin{itemize}
  \item $\int_{C} V \cdot d\vec{r}$ is path independent.
  \item $\del P / \del y = \del Q / \del x$.
  \item There is some $f$ such that $\grad f = V$.
\end{itemize}

\begin{defn}[Curl]
  The curl of a vector field $\vec{F}$ is
  \[ \curl(\vec{F}) = \grad \cross \vec{F}. \]
\end{defn}
Because curl is defined for $n \ge 2$ many variables, such as in 3 dimensions:
\[
  \curl(\veclit{P, Q, R}) = \left|\begin{matrix}[1.3]
    \ihat & \jhat & \khat\\
    \frac{\del}{\del x} & \frac{\del}{\del y} & \frac{\del}{\del z}\\
    P & Q & R
  \end{matrix}\right|
\]
which is also a vector field. For any conservative field $V$, $\curl V = \vec{0}$.

\begin{defn}[Divergence]
  The divergence of a vector field $\vec{F}$ is
  \[ \div(\vec{F}) = \grad \cdot \vec{F} = \frac{\del P}{\del x} + \frac{\del Q}{\del y} + \frac{\del R}{\del z}, \]
\end{defn}
which is simply a scalar function $f(x, y, z)$.

\begin{ex}
  What is $\div(\curl \vec{F})$?

  It is
  \[
    \frac{\del}{\del x}\parens*{\frac{\del R}{\del y} - \frac{\del Q}{\del z}}
    - \frac{\del}{\del y}\parens*{\frac{\del R}{\del x} - \frac{\del P}{\del z}}
    + \frac{\del}{\del z}\parens*{\frac{\del P}{\del y} - \frac{\del Q}{\del x}}
  \]
  which for $f \in C^{2}$ is 0, as expected (since $\curl \vec{F} = \vec{0} \implies \grad \cdot \vec{0} = 0$).
\end{ex}

\begin{theorem}
  \[
    \int_{C} \vec{F} \cdot \dx\vec{r} = \iint_{D} \curl \vec{F} \cdot \khat \dx A
  \]
  where $\vec{F} = \veclit{P, Q, 0}$.
\end{theorem}

\begin{defn}[Conservativeness]
  A vector field $\vec{F}$ is conservative if
  \[ \curl \vec{F} = 0. \]
\end{defn}

If $C$ is a curve parametrized by $\vec{r}(t) = \veclit{x(t), y(t)}$, then the unit tangent vector to $C$ is
\[ \vec{T}(t) = \frac{\vec{r}'(t)}{\norm{\vec{r}'(t)}} \]
and the unit normal vector is
\[ \vec{n}(t) = \frac{\veclit{y'(t), -x'(t)}}{\norm{\veclit{y'(t), -x'(t)}}}. \]

Then, we see
\begin{align*}
  \int_{C} \vec{F} \cdot \vec{n} \dx S &= \int_{C} \veclit{P, Q} \cdot \veclit{y'(t), -x'(t)} \, \dx t\\
  &= \int_{C} Py' - Qx' \, \dx t\\
  &= \int_{C} P \dx y - Q \dx x\\
  &= \iint_{D} \frac{\del P}{\del x} + \frac{\del Q}{\del y} \, \dx A\\
  &= \iint_{D} \div{\vec{F}} \dx A
\end{align*}

\subsection*{Surface Integrals}

Let $S$ be a 2-dimensional surface
\[ \iint_{S} \, \dx S = \quad \text{surface area of $S$} \]
and $E$ be 3-dimensional
\[ \iint_{E} \, \dx V = \quad \text{volume of $E$}. \]

Much like triple integrals, an empty integrand still has physical meaning while a function $f$ in the integrand is more
of an ``evaluation'' of $f$ over the whole surface.

Now, consider a continuous vector field $\vec{F}$ defined on an \emph{orientable} (Definition \ref{defn:orientable}) surface $S$.
The surface integral of $\vec{F}$ over $S$ is
\[ \iint_{S} \vec{F} \dx \bm{S} = \iint_{S} \vec{F} \cdot \vec{n} \dx S = \iint_{D} \div \vec{F} \dx A \]
which is sometimes called the \emph{flux} of $\vec{F}$ over $S$.

\begin{defn}[Orientable Surface]\label{defn:orientable}
  A surface is \emph{orientable} if there is a consistent sense of ``clockwise'' or ``up''/``down''.
\end{defn}

\begin{remark}
  The tangent plane to a surface $S = \veclit{x, y, z}$ at $p$ by considering the plane defined by
  the parametrization $\veclit{x(u, v), y(u, v), z(u, v)}$ we would then have
  \[ \vec{r}_{u} \times \vec{r}_{v} = \vec{n}. \]
  With $\vec{r}_{u}, \vec{r}_{v}$ the vectors defining the tangent plane, and
  $\vec{n}$ some scalar multiple of the normal vector to the tangent plane.
\end{remark}

Thus, we can say (where $D$ is the projection of $S$ into the $u,v$-plane)
\begin{align*}
  \iint\limits_{S} \vec{F} \cdot \dx \bm{S} &= \iint\limits_{D} \vec{F} \cdot (\vec{r}_{u} \times \vec{r}_{v}) \, \dx A
\end{align*}

\end{document}
